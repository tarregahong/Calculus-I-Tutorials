\documentclass{article}
\usepackage{amsmath}
\usepackage{amsthm}
\usepackage{amssymb}
\usepackage{ctex}
\usepackage{enumerate}
\usepackage{hyperref}
\usepackage{extarrows}
%\usepackage{draftwatermark}
\usepackage{titlesec,titletoc}

\begin{document}
	%\tableofcontents
	%\newpage
	\pagestyle{empty}
	\begin{center}
		{\zihao{4}\bf 习题4.1}
	\end{center}
	
		\begin{enumerate}[1.]
			\item $f(1) =0 . \quad f(-1)=0.$
			
				$f^{\prime}(x) =3 x^{2}-1$
			
				当 $\varepsilon=\pm \frac{\sqrt{3}}{3}$时. $f^{\prime}(x)=0. \therefore \varepsilon=\pm \frac{\sqrt{3}}{3}.$
			\item $f^{\prime}(x)=\frac{1}{x}$
			
				当 $\varepsilon=\frac{1}{\ln 2}$时. $f^{\prime}(\varepsilon)=\frac{f(2)-f(1)}{2-1}=\ln 2 . \quad \therefore \varepsilon=\frac{1}{\ln 2}$
				
			\item $\frac{f^{\prime}(x)}{g^{\prime}(x)}=\frac{4 x^{3}}{2 x}=2 x^{2}$
				
				当 $\varepsilon=\frac{\sqrt{10}}{2}$时. $\frac{f^{\prime}(\varepsilon)}{g(\varepsilon)}=\frac{f(2)-f(1)}{g(2)-g(1)}=5 . \quad \therefore \varepsilon=\frac{\sqrt{10}}{2}$.
				
			\item $f(x): \lim\limits_{x \rightarrow 0^{+}} \frac{f(0+\Delta x)-f(0)}{\Delta x}=\frac{\Delta x}{\Delta x}=1$
			
				$\lim\limits_{x \rightarrow 0^{-}} \frac{f(0+\Delta x)-f(0)}{\Delta x}=\frac{-\Delta x}{\Delta x}=-1$
				
				$f_{2}(x): \quad \lim\limits_{x \rightarrow 0} f_{2}(x)=\infty \neq f(0)=1$
				
				极限值$\neq$函数值 $\Longrightarrow$不连续.
				
				$f_3(x):$ 在$[0, 1]$上没有相等的两点. 
				
			\item 设$F(x)=a_{0} x+\frac{1}{2} a_{1} x^{2}+\frac{1}{3} a_{2} x^{3}+\cdots+\frac{1}{n+1} a_{n} x^{n+1}$
			
				$F(0)=F(1)=0$.
				
				由罗尔中值定理可知.
				
				$F'(x)=f(x)=a_{0}+\frac{1}{2} a_{1} x+\cdots +\frac{1}{n+1} a_{n} x^{n}$在$(0, 1)$内至少有一个零点.
			
			\item (1) 令$F(x)=\arcsin x+\arccos x$
				
				$f^{\prime}(x)=\frac{1}{\sqrt{1-x^{2}}}-\frac{1}{\sqrt{1-x^{2}}}=0$.
				
				由拉格朗日中值定理可知.
				
				$F(x)$ 在 $[-1,1]$是常数. $F(0)= \frac{\pi}{2}$
				
				$\therefore \arcsin x+\arccos x=\frac{\pi}{2}, x \in[-1,1]$.
				
				(2) 令$F(x)=3 \arccos x-\arccos \left(3 x-4 x^{3}\right)$
				
				$F^{\prime}(x)=-\frac{3}{\sqrt{1-x^{2}}}+\frac{3-12 x^{2}}{\sqrt{1-\left(3 x-4 x^{2}\right)^{2}}}=0.$
				
				由拉格朗日中值定理可知
				
				$F(x)$在$[-1,1]$内是常数. $F(0)=\pi$.
				
				$\therefore 3 \arccos x-\arccos \left(3 x-4 x^{3}\right)=\pi. \quad x \in\left[-\frac{1}{2}, \frac{1}{2}\right]$.
				
			\item (1) 当$x=y$时,等号显然成立. 设$f(x)=\sin x. f'(x)=\cos x.$ 由拉格朗日中值定理有.
			
				$\frac{\sin x-\sin y}{x-y}=\cos \varepsilon$  
				
				$\because|\cos \varepsilon| \leq 1 \quad \therefore|\sin x-\sin y| \leq|x-y| \quad, x, y \in R$.
				
				(2) 当$x=y$时,等号显然成立. 设$f(x)=\arctan x. f'(x)=\frac{1}{1+x^2}.$ 由拉格朗日中值定理有.
				
				$\frac{\arctan x-\arctan y}{x-y}=\frac{1}{1+\varepsilon^{2}}$.
				
				$\because \frac{1}{1+\varepsilon^{2}} \geqslant 1 \quad \therefore|\arctan x-\arctan y| \leqslant|x-y|$.
				
				(3) $\frac{b-a}{b}<\ln \frac{b}{a}<\frac{b-a}{a}$
				
				$\Rightarrow \frac{1}{b} <\frac{\ln b-\ln a}{b-a}<\frac{1}{a}$
				
				$f(x)=\ln x \quad(0<a \leq x \leq b) .$
				
				$f^{\prime}(x) =\frac{1}{x}$
				
				由拉格朗日中值定理, $\exists \varepsilon\in(a, b)$.
				
				使得$\frac{\ln b-\ln a}{b-a}=\frac{1}{\varepsilon}$
				
				$\because \frac{1}{b}<\frac{1}{\varepsilon}<\frac{1}{a} .$
				
				$\therefore \frac{1}{b}<\frac{\ln b-\ln a}{b-a}<\frac{1}{a} \quad$ 即 $\frac{b-a}{b}<\ln \frac{b}{a}<\frac{b-a}{a} .$
				
				(4) 题目错误,改成$nb^{n-1}(a-b)<a^n-b^n<na^{n-1}(a-b)$
				
				设$f(x)=x^{n}, \quad f^{\prime}(x)=n x^{n-1}.$
				
				由拉格朗日中值定理, $\exists \varepsilon\in(a, b)$.
				
				$\frac{f(a)-f(b)}{a-b}=f^{\prime}(\varepsilon) \quad$ 即$a^{n}-b^{n}=n \varepsilon^{n-1}(a-b)$
				
				$\therefore n b^{n-1}(a-b)<a^{n}-b^{n}<n a^{n-1}(a-b)$.
			
			\item (1) $2x[f(b)-f(a)]=\left(b^{2}-a^{2}\right) f^{\prime}(x)$.
			
				$\Leftrightarrow \frac{f(b)-f(a)}{b^{2}-a^{2}}=\frac{f^{\prime}(x)}{2 x}$
				
				令 $g(x)=x^{2} . \quad g^{\prime}(x)=2 x \neq 0, \quad x \in(a, b)$.
			
			 	由柯西中值定理. $\exists \varepsilon\in (a, b)$
			
				$\frac{f(b+f(a)}{g(b)-g(a)}=\frac{f^{\prime}(\varepsilon)}{g^{\prime}(\varepsilon)} \quad$ 即 $\frac{f(b)-f(a)}{b^{2}-a^{2}}=\frac{f^{\prime}(\varepsilon)}{2 \varepsilon}$
			
				$\therefore$ 在$(a, b)$ 内, $2 x[f(b)-f(a)]=\left(b^{2}-a^{2}\right) f^{\prime}(x)$至少存在一个实根.
			
				(2)证明:设$x_1, x_2$为$f(x)=0$的两个相异的根.
			
				设$x_1<x_2.$ 令$F(x)=e^{\alpha x}f(x)$
				
				$F'(x)=e^{\alpha x}(\alpha f(x)+f'(x))$
				
				$F(x_1)=F(x_2)=0$.
				
				由罗尔中值定理可知
				
				$f'(x)+\alpha f(x)=0$.
				
				(3) 题目错误,改成“使得$f'(x)=-f(\varepsilon)\cot \varepsilon$”.
				
				证明:令$F(x)=\sin xf(x)$
				
				$F'(x)=\sin x(f'(\varepsilon)+f(\varepsilon)\cot \varepsilon)$
				
				$F(0)=F(\pi)=0$.
				
				由罗尔中值定理可知
				
				$f'(\varepsilon)+f(\varepsilon)\cot \varepsilon=0$,
				
				即$f'(\varepsilon)=-f(\varepsilon)\cot \varepsilon$.
			
			\item 由拉格朗日中值定理有 $\frac{f(x_0+\varDelta x)-f(x_0)}{\varDelta x}=f'(\varepsilon), \varepsilon\in (x_0, x_0+\varDelta x)$.
			
				$\because f(x_0+\varDelta)-f(x_0)=f'(x_0+\theta\varDelta x)\varDelta x \qquad \therefore \varepsilon=x_0+\theta\varDelta x$.
				
				$\theta=\frac{\varepsilon-x_0}{\varDelta x}. \therefore\lim\limits_{\varDelta\to 0}\theta=\lim\limits_{\varDelta\to 0}\frac{\varepsilon-x_0}{\varDelta x}$
				
				$\because f(x)=\frac{1}{x} \quad \therefore f(x_0+\varDelta x)-f(x_0)=\frac{1}{x_0+\varDelta}-\frac{1}{x_0}=\frac{-\varDelta x}{x_0(x_0+\varDelta x}=f'(\varepsilon)\varDelta x$
				
				$\therefore f'(\varepsilon=-\frac{1}{x_0(x_0+\varDelta x)}. \quad f'(\varepsilon)=-\frac{1}{\varepsilon^2}\qquad -\frac{1}{\varepsilon^2}=-\frac{1}{x_0(x_0+\varDelta x)}$
				
				$\varepsilon=\sqrt{x_0(x_0+\varDelta x)}$ 代入$\lim\limits_{\varDelta\to 0}\frac{\varepsilon-x_0}{\varDelta x}=\frac{\sqrt{x_0(x_0+\varDelta x)}-x_0}{\varDelta x}\xlongequal[\text{洛必达}]{}\dfrac{x_0}{2\sqrt{x_0(x_0+\varDelta x)}}=\dfrac{1}{2}$.
			
			\item (1) 由拉格朗日中值定理可知, $\exists \varepsilon\in(x, x+1)$
			
				$\sqrt{x+1}-\sqrt{x}=\frac{1}{2 \sqrt{\varepsilon}}$
				
				令$\varepsilon=x+\theta(x) \quad \therefore \sqrt{x+1}-\sqrt{x}=\frac{1}{2 \sqrt{x+\theta(x)}}$
				
				化简可得$\theta(x)=\dfrac{1+2\sqrt{x(x+1)}-2x}{4}, x=0$时,$\theta(x)=\dfrac{1}{4}$.
				
				$\because 2 x<2 \sqrt{x(x+1)}<(x+1) \quad \therefore \theta(x) \in\left[\frac{1}{4}, \frac{1}{2}\right).$
				
				(2) 由(1)可知,$\theta(x)=\dfrac{1}{4}+\dfrac{1}{2}[\sqrt{x(x+1)}-x]$
				
				$\lim\limits_{x\to 0^+}\theta(x)=\dfrac{1}{4}$
				
				$\lim\limits_{x\to +\infty}\theta(x)=\frac{1}{4}+\frac{1}{2} \lim\limits _{x \rightarrow+\infty} \frac{x}{\sqrt{x(x+1)}+x}=\frac{1}{2}.$
		\end{enumerate}	
	
	\newpage
	\begin{center}
		{\zihao{4}\bf 习题4.2}
	\end{center}
	
		\begin{enumerate}[1.]
			\item 对于$\lim\limits_{x\to x_0^+}\frac{f'(x)}{g'(x)}=+\infty$或$-\infty$的情形,证明定理4.2.1.
			
				证明:由于函数在$x=x_0$处的值与$x\to x_0^+$时的极限无关.
				
				因此可以补偿定义$f(x_0)=g(x_0)=0$.
				
				这样,对任意的$x\in(x_0, x_0+\delta)$, 函数$f(t)$和$g(t)$在$[x_0, x]$上满足柯西中值定理的所有条件,故存在$\xi\in(x_0, x)$, 使得
				
				$\frac{f(x)}{g(x)}=\frac{f(x) f\left(x_{0}\right)}{g(x)-g\left(x_{0}\right)}=\frac{f^{\prime}\left(\frac{3}{3}\right)}{g^{\prime}(\xi)}$
				
				注意到,当$x\to x_0^+$时,$\xi\to x_0^+$, 故
				
				$\lim\limits_{x\to x_0^+}\frac{f(x)}{g(x)}=\lim\limits_{x\to x_0^+}\frac{f'(\xi)}{g'(\xi)}=\lim\limits_{\xi\to x_0^+}\frac{f'(\xi)}{g'(\xi)}=\lim\limits_{x\to x_0^+}\frac{f'(\xi)}{g'(\xi)}.$
				
				即证对于$\lim\limits_{x\to x_0^+}\frac{f'(x)}{g'(x)}=+\infty$或$-\infty$的情形,定理4.2.1依然成立.
			
			\item (1) $\lim\limits_{x \rightarrow 1} \frac{x^{m-1}}{x^{n}-1}(m>0, n>0)$.
			
				解:原式$=\lim\limits _{x \rightarrow 1} \frac{m \cdot x^{n-1}}{n \cdot x^{n-1}}=\frac{m}{n}$
				
				(2) $\lim\limits _{x \rightarrow 0} \frac{e^{x}-e^{-x}}{\sin x}$
				
				解:原式$=\lim\limits _{x \rightarrow 0} \frac{e^{x}+e^{-x}}{\cos x}=2.$
				
				(3)$\lim\limits_{x \rightarrow 0} \frac{\tan x-x}{x-\sin x}$.
				
				解:原式=$\lim\limits_{x \rightarrow 0} \frac{1-\cos ^2 x}{\cos^2 x(1-\cos x)}=\lim\limits_{x \rightarrow 0}\frac{1=\cos x}{\cos ^2 x}=2$.
				
				(4)$\lim\limits_{x \rightarrow 0}\frac{x^{x^2}-1}{\cos x-1}$
				
				解:原式=$\lim\limits_{x \rightarrow 0}\frac{2 x e^{x^{2}}}{-\sin x}=\lim\limits_{x \rightarrow 0} \frac{2 e^{x^{2}}+4 x^{2} e^{x^{2}}}{-\cos x}=-2$.
				
				(5)$\lim\limits_{x \rightarrow \pi}\frac{\sin 3x}{\tan 5x}$
				
				解:原式=$\lim\limits_{x \rightarrow 0}\frac{3\cos 3x}{\frac{5}{\cos^2 5x}}=\lim\limits_{x\to \pi}\frac{3\cos 3x \cdot \cos^2 5x}{5}=-\frac{3}{5}$.
				
				(6)$\lim\limits _{x \rightarrow \frac{\pi}{4}} \frac{\tan x-1}{\sin 4 x}$
				
				解:原式=$\lim\limits _{x \rightarrow \frac{\pi}{4}}\frac{1}{4\cos^2 x\cos 4x}=-\frac{1}{2}$.
				
				(7)$\lim\limits _{x \rightarrow 0} \frac{3^{x}-2^{x}}{x}$
				
				解:原式=$\lim\limits_{x \rightarrow 0}(3^x \ln 3-2^x \ln 2)=\ln 3-\ln 2=\ln \frac{3}{2}$.
				
				(8)$\lim\limits _{x \rightarrow 0} \frac{x-\arcsin x}{\sin ^{2} x}$
				
				解:原式=$\lim\limits _{x \rightarrow 0} \frac{1-\frac{1}{\sqrt{1-x^2}}}{\sin 2x}=\lim\limits _{x \rightarrow 0} \frac{-\frac{1}{2}(1-x^2)^{-\frac{3}{2}}}{2\cos 2x}=-\frac{1}{4}$
				
				(9)$\lim\limits _{x \rightarrow 0} \frac{e^{x}+\sin x-1}{\ln (1+x)}$
				
				解:原式=$\lim\limits_{x \rightarrow 0} \frac{e^{x}+\sin x-1}{x}=\lim\limits_{x \rightarrow 0}(e^x+\cos x)=2$
				
				(10)$\lim\limits _{x \rightarrow+\infty} \frac{\ln \left(1+\frac{1}{x}\right)}{\operatorname{arccot} x}$
				
				解:原式=$=\lim\limits _{x \rightarrow+\infty} \frac{-\frac{1}{x^{2}} \cdot \frac{x}{x+1}}{-\frac{1}{1+x^{2}}}=\lim\limits _{x \rightarrow+\infty} \frac{1+x^{2}}{x^{2}+x}=\lim\limits_{x \rightarrow+\infty} \frac{1+\frac{1}{x^{2}}}{1+\frac{1}{x}}=1$
				
				(11)$\lim\limits _{x \rightarrow+\infty} \frac{\ln \left(1+e^{x}\right)}{5 x}$
				
				解:原式=$\lim\limits _{x \rightarrow+\infty}\frac{e^x}{5e^x+5}=\lim\limits _{x \rightarrow+\infty}\frac{1}{5+\frac{5}{e^x}}=\frac{1}{5}$
				
				(12)$\lim\limits _{x \rightarrow+\infty} \frac{x^{2}+\ln x}{x \ln x}$
				
				解:原式=$\lim\limits _{x \rightarrow+\infty}\frac{2x+\frac{1}{x}}{\ln x+1}=\lim\limits _{x \rightarrow+\infty}\frac{2-\frac{1}{x^2}}{\frac{1}{x}}=+\infty$
				
				(13)$\lim\limits_{x \rightarrow 0^{+}}\left(\frac{1}{x}\right)^{\tan x}$
				
				解: $\because \lim\limits _{x \rightarrow 0^{+}}\left(\frac{1}{x}\right)^{\tan x}=\lim\limits _{x \rightarrow 0^{+}} e^{\tan \ln \left(\frac{1}{x}\right)}$
				
				又$ \because \lim\limits _{x \rightarrow 0^{+}} \tan x \ln \left(\frac{1}{x}\right)
				=\lim\limits _{x \rightarrow 0^{+}} \frac{-\ln x}{\cot x}
				=\lim\limits _{x \rightarrow 0^{+}} \frac{-\frac{1}{x}}{-\frac{1}{\sin ^{2} x}}
				=\lim\limits _{x \rightarrow 0^{+}} \frac{\sin ^{2} x}{x}
				=\lim\limits _{x \rightarrow 0^{+}} x=0$ 
				
				$\therefore$ 原式 $=\lim\limits _{x \rightarrow 0^{+}} e^{\tan x \ln \left(\frac{1}{x}\right)}=e^{0}=1$.
				
				(14)$\lim\limits _{x \rightarrow 0^{+}} x^{\sin x}$
				解: $\because \lim\limits _{x \rightarrow 0^{+}} x^{\sin x}=\lim\limits _{x \rightarrow 0^{+}} e^{\sin x \ln x}$.
				
				又$ \because \lim\limits _{x \rightarrow 0^{+}} \sin x \ln x
				=\lim\limits _{x \rightarrow 0^{+}} \frac{\ln x}{\sin x}
				=\lim\limits_{x \rightarrow 0^{+}} \frac{\ln x}{\frac{1}{x}}
				=\lim\limits _{x \rightarrow 0^{+}} \frac{\frac{1}{x}}{\frac{-1}{x^{2}}}
				=-\lim\limits _{x \rightarrow 0} x=0.$
				
				$\therefore$原式 $=\lim\limits _{x \rightarrow 0^{+}} e^{\sin x \ln x}=e^{0}=1$.
				
				(15)$\lim\limits _{x \rightarrow+\infty}\left(1+\frac{1}{x^{2}}\right)^{x}$
				
				解: $\because \lim\limits_{x \rightarrow+\infty}\left(1+\frac{1}{x^{2}}\right)^{x}
				=\lim\limits_{x \rightarrow+\infty} e^{x \cdot \ln \left(1+\frac{1}{x^{2}}\right)}$
				
				又$\lim\limits_{x \rightarrow+\infty} x \cdot \ln \left(1+\frac{1}{x^{2}}\right)
				=\lim\limits_{x \rightarrow+\infty} \frac{\ln \left(1+\frac{1}{x^{2}}\right)}{\frac{1}{x}}
				=\lim\limits_{x \rightarrow+\infty} \frac{-\frac{2}{x^{3}} \cdot \frac{x^{2}}{x^{2}+1}}{-\frac{1}{x^{2}}}
				=\lim\limits_{x \rightarrow+\infty}
				=\frac{2 x}{x^{2}+1}
				=\lim\limits_{x \rightarrow+\infty} \frac{2}{x+\frac{1}{x}}=0$
				
				$\therefore$原式$=\lim\limits_{x \rightarrow+\infty} e^{x \ln \left(1+\frac{1}{x^{2}}\right)}=e^{0}=1$.
				
				(16)$\lim\limits_{x\rightarrow 0}\frac{(e^{x^2}-1)\sin x^2}{x^2(1-\cos x)}$
				
				解:原式=$\frac{x^2\sin x^2}{x^2\cdot \frac{1}{2}x^2}
				=\lim\limits_{x\rightarrow 0}\frac{2\sin x^2}{x^2}
				=\lim\limits_{x\rightarrow 0}\frac{4x\cos x^2}{2x}=2$
				
				(17) $\lim\limits _{x \rightarrow 0} \frac{(1+x)^{x}-e}{x}$
				
				解:原式=$\lim\limits_{x\rightarrow 0}\frac{e^{\frac{1}{x}\ln (1+x)}-e}{x}
				=e\lim\limits_{x\rightarrow 0}\frac{e^{\frac{1}{x}\ln (1+x)-1}-1}{x}
				=e\lim\limits_{x\rightarrow 0}\frac{\frac{1}{x}\ln (1+x)-1}{x}$
				
				$=e\lim\limits_{x\rightarrow 0}\frac{\ln (1+x)-1}{x^2}
				=e\lim\limits_{x\rightarrow 0}\frac{\frac{1}{1+x}-1}{2x}
				=e\lim\limits_{x\rightarrow 0}-\frac{1}{2(1+x)}=-\frac{e}{2}$
				
				(18) $\lim\limits _{x \rightarrow 0} \frac{e^{\tan x}-e^{x}}{\tan x-x}$
				
				解:原式=$\lim\limits _{x \rightarrow 0}\frac{e^x(x^{\tan x-x}-1)}{\tan x-x}=\lim\limits_{x\rightarrow 0}\frac{e^x(\tan x-x)}{\tan x-x}=1$
				
				(19) $\lim\limits _{x \rightarrow 1} \left(\tan \frac{\pi x}{4}\right)^{\tan \frac{\pi x}{2}}$
				
				解: $\because \lim\limits _{x \rightarrow 1}\left(\tan \frac{\pi x}{4}\right)^{\tan \frac{\pi x}{2}}
				=\lim\limits _{x \rightarrow 1} e^{\tan \frac{\pi x}{2}\cdot \ln \left(\tan \frac{\pi}{4} x\right)}$.
				
				又$\because \lim\limits _{x \rightarrow 1} \tan \frac{\pi x}{2}\cdot \ln \left(\tan \frac{\pi}{4} x\right)
				=\lim\limits _{x \rightarrow 1}\frac{\ln (\tan \frac{\pi}{4}x}{\cot \frac{\pi}{2}x}
				=\lim\limits _{x \rightarrow 1}
				\frac{\frac{1}{\tan \frac{\pi}{4}x}\cdot \frac{\frac{\pi}{4}}{\cos ^2\frac{\pi}{4}x}}{-\frac{\frac{\pi}{2}}{\sin ^2\frac{\pi}{2}x}}$
				
				$=-\lim\limits _{x \rightarrow 1} \sin \frac{\pi}{2}x=-1$
				
				$\therefore$ 原式 $=\lim\limits _{x \rightarrow 1}e^{\tan \frac{\pi x}{2}\cdot \ln \left(\tan \frac{\pi}{4} x\right)}=e^{-1}=\frac{1}{e}$
				
				(20) $\lim\limits _{x \rightarrow 0}\left(\frac{2}{\pi} \arccos x\right)^{\frac{1}{x}}$
				
				解:$\because \lim\limits _{x \rightarrow 0} (\frac{\frac{2}{\pi} \arccos x})^{\frac{1}{x}}
				=\lim\limits _{x \rightarrow 0}e^{\frac{\ln \frac{2}{\pi}\arccos x}{x}}$
				
				又 $\because \lim\limits _{x \rightarrow 0}\frac{\ln \frac{2}{\pi}\arccos x}{x}
				=\lim\limits _{x \rightarrow 0} \frac{1}{\frac{2}{\pi}\arccos x} \cdot \frac{-\frac{2}{\pi}}{\sqrt{1-x^{2}}}
				=\lim\limits _{x \rightarrow 0}-\frac{1}{\arccos x \cdot \sqrt{1-x^2}}=-\frac{2}{\pi}$
				
				$\because$ 原式 $=\lim\limits _{x \rightarrow 0}e^{\frac{\ln \frac{2}{\pi}\arccos x}{x}}=e^{-\frac{2}{\pi}}$
				
				(21) $\lim\limits _{x \rightarrow 1^{-}} \ln x \ln (1-x)$
				
				解:原式 $=\lim\limits _{x \rightarrow 1^{-}}\frac{\ln (1-x)}{\frac{1}{\ln x}}
				=\lim\limits _{x \rightarrow 1^{-}}\frac{x\ln^2 x}{1-x}
				=\lim\limits _{x \rightarrow 1^{-}}\frac{\ln^2 x+2\ln x}{-1}=0$
				
				(22) $\lim\limits _{x \rightarrow 0}\left((1+x)^{\frac{1}{x}} / e\right)^{\frac{1}{x}}$ 
				
				解:原式$=\lim\limits _{x \rightarrow 0} e^{\frac{1}{x}\ln [(1+x)^\frac{1}{x}/e]}
				=\lim\limits _{x \rightarrow 0} e^{\frac{1}{x}[\frac{1}{x}\ln (1+x)-1]}
				=\lim\limits _{x \rightarrow 0}e^{\frac{\ln(1+x)-x}{x^2}}$
				
				$=\lim\limits _{x \rightarrow 0} e^{\frac{\frac{1}{1+x}-1}{2x}}
				=\lim\limits _{x \rightarrow 0} e^{-\frac{1}{2(1+x)}}
				=e^{-\frac{1}{2}}$
				
				(23) $\lim\limits _{x \rightarrow 0}\left(\cot x-\frac{1}{x}\right)$
				
				解:原式 $=\lim\limits _{x \rightarrow 0}\frac{x\cos x-\sin x}{x\sin x}
				=\lim\limits _{x \rightarrow 0}\frac{-x\sin x}{\sin x+x\cos x}
				=\lim\limits _{x \rightarrow 0}\frac{-\sin x-x \cos x}{2\cos x-x\sin x}=0$
				
				或原式 $=\lim\limits _{x \rightarrow 0}\left(\frac{1}{\tan x}-\frac{1}{x}\right)
				=\lim\limits _{x \rightarrow 0} \frac{x-\tan x}{x\tan x}
				=\lim\limits _{x \rightarrow 0} \frac{x-\tan x}{x^{2}}
				=\lim\limits _{x \rightarrow 0} \frac{1-\sec ^{2} x}{2 x}
				=\lim\limits _{x \rightarrow 0} \frac{-2 \sec ^{2} x \tan x}{2}=0$
				
				(24) $\lim\limits _{x \rightarrow 0^{+}}\left(\frac{1}{m}\left(a_{1}^{x}+a_{2}^{x}+\cdots+a_{m}^{x}\right)^{\frac{1}{x}}\left(a_{1}, a_{2}, \ldots, a_{m}>0\right)\right.$
				
				解:原式$=\lim\limits_{x\rightarrow 0^{+}}e^{\frac{\ln \frac{a_{1}^{x}+a_{2}^{x}+\cdots+a_{m}^{x}}{m}}{x}}$
								
				$\therefore \lim\limits _{x \rightarrow 0^{+}} \frac{\ln \frac{a_{1} x+a_2^{x}+\cdots+a_{m}^{x}}{m}}{x}
				=\lim\limits _{x \rightarrow 0^{+}} \frac{m}{a_{1}^{x}+a_{2}^{x}+\cdots+a_{m}^{x}} \cdot \frac{1}{m}\left(a_{1}^{x} \ln a_{1}+a_{2}^{x} \ln a_{2}+\cdots +a_{m}^{x} \ln a_{m}\right)$
					
				$=\frac{1}{m}\left(\ln a_{1}+\ln a_{2}+\cdots+\ln a_{m}\right)=\ln \left(a_{1} a_{2} \cdots a_{m}\right)^{\frac{1}{m}}$
					
				$\therefore$原式 $=e^{\ln \left(a_{1} a_{2} \cdots a_{m}\right)^{\frac{1}{m}}}=\left(a_{1} a_{2} \cdots a_{m}\right)^{\frac{1}{m}}$
				
			\item 说明不能用洛必达法则求下列极限
			
				(1)$\lim\limits_{x\rightarrow +\infty}\frac{x+\sin x}{x-\sin x}$
				
				解:当$x\rightarrow +\infty$时,$(\frac{x+\sin x}{x-\sin x})^{'} =\frac{1+\cos x}{1-\cos x}$极限不存在.
				
				故$\lim\limits_{x\rightarrow +\infty}\frac{x+\sin x}{x-\sin x}$不能用洛必达法则求极限.
				
				(2)$\lim\limits _{x \rightarrow 0} \frac{x^{2} \sin \frac{1}{x}}{\sin x}$
				
				解:当$x\rightarrow 0$时,$(\frac{x^2\sin \frac{1}{x}}{\sin x})^{'}=\frac{2x\frac{1}{x}-\cos \frac{1}{x}}{\cos x}$极限不存在.
				
				故$\lim\limits _{x \rightarrow 0} \frac{x^{2} \sin \frac{1}{x}}{\sin x}$不能用洛必达法则求极限.
		\end{enumerate}
\end{document}